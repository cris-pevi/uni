\documentclass{article}
\usepackage{amsmath}

\begin{document}
\large % Cambia el tamaño del texto a "large"

\section*{Problema 01.}

\subsection*{Sea \(f(x)\) una antiderivada de \(v(x)\). Determine el valor de verdad de las siguientes proposiciones.}

\subsection*{\newline a) \(3f(x)\) es una antiderivada de \(3v(x)\)}
\subsection*{Solución: } 

Dado que \(f(x)\) es una antiderivada de \(v(x)\), es decir:
\[
f'(x) = v(x),
\]

\noindent Debemos verificar si \(3f(x)\) es una antiderivada de \(3v(x)\).

\noindent \newline Consideremos la función \(F(x) = 3f(x)\) y calculemos su derivada:
\[
F'(x) = \frac{d}{dx}[3f(x)]
\]

\noindent \newline Usando la regla de la constante multiplicada por una función, tenemos:
\[
F'(x) = 3f'(x)
\]

\noindent Dado que \(f'(x) = v(x)\), podemos sustituir:
\[
F'(x) = 3v(x)
\]

\noindent Por lo tanto, la derivada de \(3f(x)\) es \(3v(x)\), lo que demuestra que \(3f(x)\) es efectivamente una antiderivada de \(3v(x)\).

\subsection*{Conclusión:}

La proposición \textbf{(a)} es \textbf{verdadera}.

\end{document}