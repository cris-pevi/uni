\documentclass{article}
\usepackage{amsmath}

\begin{document}
\large % Cambia el tamaño del texto a "large"

\section*{Problema 01.}

\subsection*{Sea \(f(x)\) una antiderivada de \(v(x)\). Determine el valor de verdad de las siguientes proposiciones.}

\subsection*{\newline a. \(3f(x)\) es una antiderivada de \(3v(x)\)}
\subsection*{Solución: } 

Dado que \(f(x)\) es una antiderivada de \(v(x)\), es decir:
\[
f'(x) = v(x),
\]

\noindent Debemos verificar si \(3f(x)\) es una antiderivada de \(3v(x)\).

\noindent \newline Consideremos la función \(F(x) = 3f(x)\) y calculemos su derivada:
\[
F'(x) = \frac{d}{dx}[3f(x)]
\]

\noindent \newline Usando la regla de la constante multiplicada por una función, tenemos:
\[
F'(x) = 3f'(x)
\]

\noindent Dado que \(f'(x) = v(x)\), podemos sustituir:
\[
F'(x) = 3v(x)
\]

\noindent Por lo tanto, la derivada de \(3f(x)\) es \(3v(x)\), lo que demuestra que \(3f(x)\) es efectivamente una antiderivada de \(3v(x)\).

\subsection*{Conclusión:}

La proposición \textbf{(a)} es \textbf{verdadera}.


\subsection*{\newline b. \(2f(2x)\) es una antiderivada de \(v(2x)\)}
\subsection*{Solución: } 

Dado que \(f(x)\) es una antiderivada de \(v(x)\), es decir:
\[
f'(x) = v(x),
\]

\noindent Consideremos la función \(G(x) = 2f(2x)\) y calculemos su derivada:
\[
G'(x) = \frac{d}{dx}[2f(2x)]
\]

\noindent Usando la regla de la cadena, tenemos:
\[
G'(x) = 2 \cdot f'(2x) \cdot 2 = 4f'(2x)
\]

\noindent Dado que \(f'(2x) = v(2x)\), podemos sustituir:
\[
G'(x) = 4v(2x)
\]

\noindent Por lo tanto, la derivada de \(2f(2x)\) es \(4v(2x)\), no \(v(2x)\).

\subsection*{Conclusión:}

La proposición \textbf{(b)} es \textbf{falsa}.

\subsection*{\newline c. \(f(x + 3)\) es una antiderivada de \(v(x + 3)\)}
\subsection*{Solución: } 

Dado que \(f(x)\) es una antiderivada de \(v(x)\), es decir:
\[
f'(x) = v(x),
\]

\noindent Consideremos la función \(H(x) = f(x + 3)\) y calculemos su derivada:
\[
H'(x) = \frac{d}{dx}[f(x + 3)]
\]

\noindent Usando la regla de la cadena, tenemos:
\[
H'(x) = f'(x + 3)
\]

\noindent Dado que \(f'(x + 3) = v(x + 3)\), podemos concluir:
\[
H'(x) = v(x + 3)
\]

\noindent Por lo tanto, la derivada de \(f(x + 3)\) es \(v(x + 3)\), lo que demuestra que \(f(x + 3)\) es efectivamente una antiderivada de \(v(x + 3)\)

\subsection*{Conclusión:}

La proposición \textbf{(c)} es \textbf{verdadera}.

\section*{Problema 02.}

\subsection*{Determine la antiderivada \( F \) de la función \( f \), suponiendo que \( F(0) = 0 \).}

\subsection*{\newline a. \(f(x) = 4x - \sqrt{x}\)}
\subsection*{Solución: }

La antiderivada de \( f(x) \) es:
\[
F(x) = \int (4x - \sqrt{x}) \, dx
\]

\noindent Calculamos cada término por separado:
\[
\int 4x \, dx = 2x^2, \quad \int x^{1/2} \, dx = \frac{2}{3}x^{3/2}
\]

\noindent Combinando ambos términos:
\[
F(x) = 2x^2 - \frac{2}{3}x^{3/2} + C
\]

\noindent Usando la condición inicial \( F(0) = 0 \):
\[
C = 0
\]

\noindent La antiderivada es:
\[
F(x) = 2x^2 - \frac{2}{3}x^{3/2}
\]

\subsection*{Conclusión:}

La antiderivada de \( f(x) = 4x - \sqrt{x} \) es \( F(x) = 2x^2 - \frac{2}{3}x^{3/2} \)

\subsection*{\newline b. \(f(x) = \frac{1}{(x+1)^2}\)}
\subsection*{Solución: }

La antiderivada de \( f(x) \) es:
\[
F(x) = \int \frac{1}{(x+1)^2} \, dx = -\frac{1}{x+1} + C
\]

\noindent Usando la condición inicial \( F(0) = 0 \):
\[
C = 1
\]

\noindent La antiderivada es:
\[
F(x) = -\frac{1}{x+1} + 1
\]

\subsection*{Conclusión:}

La antiderivada de \( f(x) = \frac{1}{(x+1)^2} \) es \( F(x) = -\frac{1}{x+1} + 1 \)

\subsection*{\newline c. \(f(x) = e^{-2x} + 3x^2\)}
\subsection*{Solución: }

La antiderivada de \( f(x) \) es:
\[
F(x) = \int (e^{-2x} + 3x^2) \, dx = -\frac{1}{2}e^{-2x} + x^3 + C
\]

\noindent Usando la condición inicial \( F(0) = 0 \):
\[
C = \frac{1}{2}
\]

\noindent La antiderivada es:
\[
F(x) = -\frac{1}{2}e^{-2x} + x^3 + \frac{1}{2}
\]

\subsection*{Conclusión:}

La antiderivada de \( f(x) = e^{-2x} + 3x^2 \) es \( F(x) = -\frac{1}{2}e^{-2x} + x^3 + \frac{1}{2} \).


\end{document}