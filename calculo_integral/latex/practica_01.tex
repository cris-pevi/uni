\documentclass{article}
\usepackage{amsmath}

\begin{document}
\large % Cambia el tamaño del texto a "large"

\section*{Problema 01.}

\subsection*{Sea \(f(x)\) una antiderivada de \(v(x)\). Determine el valor de verdad de las siguientes proposiciones.}

\subsection*{\newline a. \(3f(x)\) es una antiderivada de \(3v(x)\)}
\subsection*{Solución: } 

Dado que \(f(x)\) es una antiderivada de \(v(x)\), es decir:
\[
f'(x) = v(x),
\]

\noindent Debemos verificar si \(3f(x)\) es una antiderivada de \(3v(x)\).

\noindent \newline Consideremos la función \(F(x) = 3f(x)\) y calculemos su derivada:
\[
F'(x) = \frac{d}{dx}[3f(x)]
\]

\noindent \newline Usando la regla de la constante multiplicada por una función, tenemos:
\[
F'(x) = 3f'(x)
\]

\noindent Dado que \(f'(x) = v(x)\), podemos sustituir:
\[
F'(x) = 3v(x)
\]

\noindent Por lo tanto, la derivada de \(3f(x)\) es \(3v(x)\), lo que demuestra que \(3f(x)\) es efectivamente una antiderivada de \(3v(x)\).

\subsection*{Conclusión:}

La proposición \textbf{(a)} es \textbf{verdadera}.


\subsection*{\newline b. \(2f(2x)\) es una antiderivada de \(v(2x)\)}
\subsection*{Solución: } 

Dado que \(f(x)\) es una antiderivada de \(v(x)\), es decir:
\[
f'(x) = v(x),
\]

\noindent Consideremos la función \(G(x) = 2f(2x)\) y calculemos su derivada:
\[
G'(x) = \frac{d}{dx}[2f(2x)]
\]

\noindent Usando la regla de la cadena, tenemos:
\[
G'(x) = 2 \cdot f'(2x) \cdot 2 = 4f'(2x)
\]

\noindent Dado que \(f'(2x) = v(2x)\), podemos sustituir:
\[
G'(x) = 4v(2x)
\]

\noindent Por lo tanto, la derivada de \(2f(2x)\) es \(4v(2x)\), no \(v(2x)\).

\subsection*{Conclusión:}

La proposición \textbf{(b)} es \textbf{falsa}.

\subsection*{\newline c. \(f(x + 3)\) es una antiderivada de \(v(x + 3)\)}
\subsection*{Solución: } 

Dado que \(f(x)\) es una antiderivada de \(v(x)\), es decir:
\[
f'(x) = v(x),
\]

\noindent Consideremos la función \(H(x) = f(x + 3)\) y calculemos su derivada:
\[
H'(x) = \frac{d}{dx}[f(x + 3)]
\]

\noindent Usando la regla de la cadena, tenemos:
\[
H'(x) = f'(x + 3)
\]

\noindent Dado que \(f'(x + 3) = v(x + 3)\), podemos concluir:
\[
H'(x) = v(x + 3)
\]

\noindent Por lo tanto, la derivada de \(f(x + 3)\) es \(v(x + 3)\), lo que demuestra que \(f(x + 3)\) es efectivamente una antiderivada de \(v(x + 3)\)

\subsection*{Conclusión:}

La proposición \textbf{(c)} es \textbf{verdadera}.

\section*{Problema 02.}

\subsection*{Determine la antiderivada \( F \) de la función \( f \), suponiendo que \( F(0) = 0 \).}

\subsection*{\newline a. \(f(x) = 4x - \sqrt{x}\)}
\subsection*{Solución: }

La antiderivada de \( f(x) \) es:
\[
F(x) = \int (4x - \sqrt{x}) \, dx
\]

\noindent Calculamos cada término por separado:
\[
\int 4x \, dx = 2x^2, \quad \int x^{1/2} \, dx = \frac{2}{3}x^{3/2}
\]

\noindent Combinando ambos términos:
\[
F(x) = 2x^2 - \frac{2}{3}x^{3/2} + C
\]

\noindent Usando la condición inicial \( F(0) = 0 \):
\[
C = 0
\]

\noindent La antiderivada es:
\[
F(x) = 2x^2 - \frac{2}{3}x^{3/2}
\]

\subsection*{Conclusión:}

La antiderivada de \( f(x) = 4x - \sqrt{x} \) es \( F(x) = 2x^2 - \frac{2}{3}x^{3/2} \)

\subsection*{\newline b. \(f(x) = \frac{1}{(x+1)^2}\)}
\subsection*{Solución: }

La antiderivada de \( f(x) \) es:
\[
F(x) = \int \frac{1}{(x+1)^2} \, dx = -\frac{1}{x+1} + C
\]

\noindent Usando la condición inicial \( F(0) = 0 \):
\[
C = 1
\]

\noindent La antiderivada es:
\[
F(x) = -\frac{1}{x+1} + 1
\]

\subsection*{Conclusión:}

La antiderivada de \( f(x) = \frac{1}{(x+1)^2} \) es \( F(x) = -\frac{1}{x+1} + 1 \)

\subsection*{\newline c. \(f(x) = e^{-2x} + 3x^2\)}
\subsection*{Solución: }

La antiderivada de \( f(x) \) es:
\[
F(x) = \int (e^{-2x} + 3x^2) \, dx = -\frac{1}{2}e^{-2x} + x^3 + C
\]

\noindent Usando la condición inicial \( F(0) = 0 \):
\[
C = \frac{1}{2}
\]

\noindent La antiderivada es:
\[
F(x) = -\frac{1}{2}e^{-2x} + x^3 + \frac{1}{2}
\]

\subsection*{Conclusión:}

La antiderivada de \( f(x) = e^{-2x} + 3x^2 \) es \( F(x) = -\frac{1}{2}e^{-2x} + x^3 + \frac{1}{2} \).

\section*{Problema 03.}

\subsection*{Determine las integrales indefinidas de las siguientes funciones:}

\subsection*{\newline a. \(f(x) = \frac{1}{(x+2)\sqrt{x}}\)}
\subsection*{Solución: }

Para resolver esta integral, utilizamos la sustitución \( u = \sqrt{x} \), de modo que \( x = u^2 \) y \( dx = 2u \, du \).

La integral se convierte en:
\[
\int \frac{1}{(u^2 + 2)u} \cdot 2u \, du = 2 \int \frac{1}{u^2 + 2} \, du
\]

Usando la fórmula para la integral de \( \frac{1}{u^2 + a^2} \), donde \( a = \sqrt{2} \):
\[
2 \int \frac{1}{u^2 + (\sqrt{2})^2} \, du = 2 \cdot \frac{1}{\sqrt{2}} \arctan\left( \frac{u}{\sqrt{2}} \right) + C
\]

Sustituyendo \( u = \sqrt{x} \), obtenemos:
\[
F(x) = \sqrt{2} \arctan\left( \frac{\sqrt{x}}{\sqrt{2}} \right) + C
\]

\subsection*{Conclusión:}

La integral indefinida de \( f(x) = \frac{1}{(x+2)\sqrt{x}} \) es:
\[
F(x) = \sqrt{2} \arctan\left( \frac{\sqrt{x}}{\sqrt{2}} \right) + C
\]

\subsection*{\newline b. \(f(x) = \frac{x}{\sqrt{x+3} + 4}\)}
\subsection*{Solución: }

Para resolver esta integral, usamos la sustitución \( u = \sqrt{x+3} \), de modo que \( x = u^2 - 3 \) y \( dx = 2u \, du \).

\noindent \newline Reescribimos la integral en términos de \( u \):

\[
\int \frac{u^2 - 3}{u + 4} \cdot 2u \, du = 2 \int \frac{u(u^2 - 3)}{u + 4} \, du = 2 \int \frac{u^3 - 3u}{u + 4} \, du
\]

\noindent Dividimos el numerador por el denominador:

\[
= 2 \int \left( u^2 - 4u + \frac{16}{u+4} \right) \, du
\]

\noindent Ahora integramos término a término:

\[
\int u^2 \, du = \frac{u^3}{3}
\]

\[
\int -4u \, du = -2u^2\
\]

\[
\int \frac{16}{u+4} \, du = 16 \ln|u+4|
\]

\noindent Por lo tanto:
\[
F(u) = \frac{2}{3}u^3 - 4u^2 + 32 \ln|u+4| + C
\]

\noindent Sustituyendo \( u = \sqrt{x+3} \):
\[
F(x) = \frac{2}{3}(\sqrt{x+3})^3 - 4(x+3) + 32 \ln|\sqrt{x+3} + 4| + C
\]

\noindent Simplificando:
\[
F(x) = \frac{2}{3}(x+3)^{3/2} - 4(x+3) + 32 \ln|\sqrt{x+3} + 4| + C
\]

\subsection*{Conclusión:}

La integral indefinida de \( f(x) = \frac{x}{\sqrt{x+3} + 4} \) es:
\[
F(x) = \frac{2}{3}(x+3)^{3/2} - 4(x+3) + 32 \ln|\sqrt{x+3} + 4| + C
\]

\subsection*{\newline c. \(f(x) = \frac{x^3}{\sqrt{1+x^2} + 1}\)}
\subsection*{Solución: }

Para resolver esta integral, utilizamos la sustitución \( u = \sqrt{1+x^2} \), de modo que:
\[
x = \sqrt{u^2 - 1},  dx = \frac{u \, du}{\sqrt{u^2 - 1}}
\]

\noindent \newline Reescribimos la integral en términos de \( u \):
\[
\int \frac{(\sqrt{u^2 - 1})^3}{u + 1} \cdot \frac{u \, du}{\sqrt{u^2 - 1}} = \int \frac{u^4 - 2u^2 + 1}{(u + 1)^2} \, du
\]

\noindent Simplificamos dividiendo:
\[
\int \left( u^2 - 2 + \frac{3}{u+1} \right) \, du
\]

\noindent Ahora integramos término a término:
\[
\int u^2 \, du = \frac{u^3}{3}
\]

\[
\int -2 \, du = -2u
\]

\[
\int \frac{3}{u+1} \, du = 3 \ln|u+1|
\]

\noindent Por lo tanto:

\[
F(u) = \frac{u^3}{3} - 2u + 3 \ln|u+1| + C
\]

\noindent \newline Sustituyendo \( u = \sqrt{1+x^2} \):

\[
F(x) = \frac{(1+x^2)^{3/2}}{3} - 2\sqrt{1+x^2} + 3 \ln|\sqrt{1+x^2} + 1| + C
\]

\subsection*{Conclusión:}

La integral indefinida de \( f(x) = \frac{x^3}{\sqrt{1+x^2} + 1} \) es:
\[
F(x) = \frac{(1+x^2)^{3/2}}{3} - 2\sqrt{1+x^2} + 3 \ln|\sqrt{1+x^2} + 1| + C
\]

\subsection*{\newline d. \(f(x) = e^x \left( \frac{1}{x} - \frac{1}{x^2} \right)\)}
\subsection*{Solución: }

Distribuimos y reescribimos la integral:

\[
\int e^x \left( \frac{1}{x} - \frac{1}{x^2} \right) \, dx = \int \frac{e^x}{x} \, dx - \int \frac{e^x}{x^2} \, dx
\]

\noindent \newline Para la segunda integral, utilizamos integración por partes. Definimos: \( u = \frac{1}{x} \), \( dv = e^x \, dx \), entonces \( du = -\frac{1}{x^2} \, dx \), \( v = e^x \).

\noindent \newline Aplicando la integración por partes:

\[
\int \frac{e^x}{x^2} \, dx = \frac{e^x}{x} + \int \frac{e^x}{x} \, dx
\]

\noindent Por lo tanto, la integral total es:

\[
F(x) = \frac{e^x}{x} + \text{Ei}(x) + C
\]

\subsection*{Conclusión:}

La integral indefinida de \( f(x) = e^x \left( \frac{1}{x} - \frac{1}{x^2} \right) \) es:
\[
F(x) = \frac{e^x}{x} + \text{Ei}(x) + C
\]


\end{document}